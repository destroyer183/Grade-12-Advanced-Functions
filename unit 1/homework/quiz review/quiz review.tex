\documentclass[14pt, a4paper]{extarticle}
\title{Quiz Review}
\usepackage[a4paper, papersize={8.5in, 120in}, total={7.5in, 120in}]{geometry}
\usepackage{setspace}
\usepackage{amsmath}
\usepackage{amssymb}
\usepackage{changepage}
\usepackage{tikz}
\usepackage{pgfplots}
\usepackage{polynom}

\author{Jacob Zante}
\date{September 20th, 2024}
\doublespacing
\pgfplotsset{width=7in, compat=newest}
% \usepgfplotslibrary{external}

\begin{document}
\maketitle
\setlength{\parindent}{0pt}

\textbf{1. When $x^3 - kx^2 - 10kx + 25$ is divided by $x - 2$, the remainder is $9$. 
Find the value of k.}
\begin{adjustwidth}{1cm}{0pt}
    as per the remainder theorem, if $f(x)$ is divided by $x - b$, then the remainder
    is $f(b)$. with this, we know that $f(2) = 9$, and we can use this to solve for $k$.
    \begin{math}
        f(x) = x^3 - kx^2 - 10kx + 25 \\
        f(2) = (2)^3 - k(2)^2 - 10k(2) + 25 \\
        9 = 8 - 4k - 20k + 25 \\
        9 = -24k + 33 \\
        24k = 24 \\
        k = 1 \\
    \end{math}
\end{adjustwidth}

\textbf{2. If $2x^3 -9x^2 + 13x + k$ is divisible by $2x - 1$, then it is also divisible by} \\
\begin{adjustwidth}{1cm}{0pt}
    \textbf{(A) $x - 2$ \quad (B) $x - 1$ \quad (C) $x + \frac{1}{2}$ \quad (D) $x - \frac{1}{2}$ \quad (E) $2x + 1$} \\
    \\
    if $f(x) = 2x^3 - 9x^2 + 13x + k$ is divisible by $2x - 1$, 
    then as per the factor theorem, it means $f(\frac{1}{2}) = 0$, 
    and using this, we can solve for $k$. \\
    \begin{math}
        f(\frac{1}{2}) = 2(\frac{1}{2})^3 - 9(\frac{1}{2})^2 + 13(\frac{1}{2}) + k \\
        0 = 2(\frac{1}{2})^3 - 9(\frac{1}{2})^2 + 13(\frac{1}{2}) + k \\
        0 = 2(\frac{1}{8}) - 9(\frac{1}{4}) + 13(\frac{1}{2}) + k \\
        0 = \frac{1}{4} - \frac{9}{4} + \frac{13}{2} + k \\
        0 = \frac{-8}{4} + \frac{26}{4} + k \\
        0 = \frac{18}{4} + k \\
        k = \frac{-9}{2} \\
        f(x) = 2x^3 - 9x^2 + 13x - \frac{9}{2} \\
        f2(x) = 2(2x^3 - 9x^2 + 13x - \frac{9}{2}) \\
        f2(x) = 4x^3 - 18x^2 + 26x - 9 \\
        \\
    \end{math}
    now that we have the value of $k$, we can determine values of x that will create 
    another divisor, which can be done by using the factors of the first term of 
    the equation as denominators, and the factors of the last term of the equation 
    as numerators. This creates these possibilities:
    \begin{center}
        \begin{singlespace}
            \underline{$\pm$(1, 3, 9)} \\
            $\pm$(1, 2, 4) \\
        \end{singlespace}
    \end{center}
    with this, we can create multiple fractions, and substitute them into $f(x)$ 
    until one of them produces no remainder. \\
    \\
    \begin{math}
        f(\frac{1}{1}) = 2(1)^3 - 9(1)^2 + 13(1) - \frac{9}{2} = \frac{3}{2} \\
        f(\frac{-1}{1}) = 2(-1)^3 - 9(-1)^2 + 13(-1) - \frac{9}{2} = \frac{-57}{2} \\
        f(\frac{3}{1}) = 2(3)^3 - 9(3)^2 + 13(3) - \frac{9}{2} = \frac{23}{2} \\
        f(\frac{1}{4}) = 2(\frac{1}{4})^3 - 9(\frac{1}{4})^2 - 13(\frac{1}{4}) - \frac{9}{2} = -1.78125 \\
        f(\frac{-1}{4}) = 2(\frac{-1}{4})^3 - 9(\frac{-1}{4})^4 - 13 (\frac{-1}{4}) - \frac{9}{2} = -8.34375 \\
    \end{math}
    or maybe I could just relalize that $2x - 1 = x - \frac{1}{2}$ which is answer D. \\
\end{adjustwidth}

\textbf{3. If one root of the equation $x^3 - 5x^2 + 5x - 1 = 0$ is $2 - \sqrt{3}$, 
then find the sum of the other two roots.}
\begin{adjustwidth}{1cm}{0pt}
    % test
    % $ \polylongdiv{x^3 - 5x^2 + 5x - 1}{x - 2 - 3^0.5} $ \\
    % $ \polylongdiv{x^4 - 16x^3 + 4x^2 + 10x - 11}{x - 1} $ \\
\end{adjustwidth}
\textbf{4. if m, n and 1 are non-zero roots of the equation $x^3 - mx^2 + nx - 1 = 0$, 
then find the sum of the roots.}
\begin{adjustwidth}{1cm}{0pt}
\end{adjustwidth}

\textbf{5. The remainder when $f(x) = x^5 - 2x^4 + ax^3 - x^2 + bx - 2$ is divided by $x + 1$ is $-7$. 
When $f(x)$ is divided by $x - 2$, the remainder is $32$. 
Determine the remainder when $f(x)$ is divided by $x - 1$.}
\begin{adjustwidth}{1cm}{0pt}
    \begin{math}
        f(-1) = (-1)^5 - 2(-1)^4 + a(-1)^3 - (-1)^2 + b(-1) - 2 \\
        -7 = -1 - 2 - a - 1 - b - 2 \\
        -7 = -6 - a - b \\
        -1 = - a - b \\
        a = 1 - b \\
        \\
        f(2) = (2)^5 - 2(2)^4 + a(2)^3 - (2)^2 + b(2) - 2 \\
        32 = 32 - 32 + 8a - 4 + 2b - 2 \\
        32 = -6 + 8a + 2b \\
        19 = 4a + b \\
        \\
        19 = 4(1 - b) + b \\
        19 = 4 - 4b + b \\
        15 = -3b \\
        b = -5 \\
        \\
        a = 1 - b \\
        a = 1 - (-5) \\
        a = 1 + 5 \\
        a = 6 \\
        \\
        f(x) = x^5 - 2x^4 + (6)x^3 - x^2 + (-5)x - 2 \\
        f(x) = x^5 - 2x^4 + 6x^3 - x^2 - 5x - 2 \\
        \\
        f(1) = (1)^5 - 2(1)^4 + 6(1)^3 - (1)^2 - 5(1) - 2 \\
        f(1) = 5 - 2 + 6 - 1 - 5 - 2 \\
        f(1) = 1 \\
        \therefore \text{the remainder is $1$ when $f(x)$ is divided by $x - 1$.} \\
    \end{math}
\end{adjustwidth}

\textbf{6. For the polynomial $f(x) = ax^3 + bx^2 + cx + d$, 
the sum of the coefficients is equal to zero. 
(i.e. $a + b + c + d = 0$).}
\begin{adjustwidth}{1cm}{0pt}
    \textbf{a) Show that the polynomial is divisible by $x - 1$.} \\
    as per the remainder theorem, if $f(x)$ is to be divided by $x - b$, the remainder 
    will always be $f(b)$. In this case, $b$ is 1, so the remainder is equal to $f(1)$. \\
    when $1$ is inputted for $x$, all of the coefficients stay the same because $1^n = 1$, 
    meaning that the remainder would be equal to the sum of the coefficients
    of the equation, and so the remainder is 0 when $f(x)$ is divided by $x - 1$. \\
    Because of this, as per the factor theorem, we know that $x - b$ is a factor of $f(x)$ 
    iff $f(b) = 0$, and in this case that is true, meaning that $x - 1$ is a factor of 
    $f(x) = ax^3 + bx^2 + cx + d$.
    \\
    \textbf{b) Solve $2x^3 - 3x^2 + 1 = 0$.} \\
    \\
    \textbf{c) Find the sum of the coefficients in the expansion of $g(x) = (x + 2)(x^2 + 2x + 1)^2$} \\
\end{adjustwidth}

\textbf{7. Find $a$ and $b$ so that the quartic function $f(x) = a^2x^4 + 3x^3 + b^2x^2 + 4abx + 4ab$ 
leaves a remainder of $10$ on division by $x + 1$ and a remainder of $20$ on division by $x$.}
\begin{adjustwidth}{1cm}{0pt}
\end{adjustwidth}

\textbf{8. An unknown polynomial f(x) of degree $37$ yields a remainder of $1$ when divided by $x - 1$, 
a remainder of $3$ when divided by $x - 3$, and a remainder of $21$ when divided by $x - 5$. 
Find the remainder when $f(x)$ is divided by $(x - 1)(x - 3)(x - 5)$.}
\begin{adjustwidth}{1cm}{0pt}
\end{adjustwidth}

\textbf{9. if $ax^3 + bx + c$, with $a \neq 0, c \neq 0$, has a factor of the form $x^2 + px + 1$, 
show that $a^2 - c^2 = ab$.}
\begin{adjustwidth}{1cm}{0pt}
\end{adjustwidth}

\textbf{10. Given that the cubic equation $x^3 - 3x^2 + ax + b = 0$ has rational coefficients and has the root $-1\emph{i}\sqrt{3}$, 
determine the values of a and b.}
\begin{adjustwidth}{1cm}{0pt}
\end{adjustwidth}

\end{document}