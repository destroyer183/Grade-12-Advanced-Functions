\documentclass[14pt, a4paper]{extarticle}
\title{1.10 Even/Odd Functions and Symmetry}
\usepackage[a4paper, papersize={8.5in, 120in}, total={7.5in, 120in}]{geometry}
\usepackage{setspace}
\usepackage{amsmath}
\usepackage{amssymb}
\usepackage{changepage}
\usepackage{tikz}
\usepackage{pgfplots}
\usepackage{polynom}

\author{Jacob Zante}
\date{October 1st, 2024}
\doublespacing
\pgfplotsset{width=7in, compat=newest}
% \usepgfplotslibrary{external}

\begin{document}
\maketitle
\setlength{\parindent}{0pt}

\textbf{Determine whether each of the functions below is even, odd, or neither. Justify your answers.} \\
\begin{adjustwidth}{1cm}{0pt}
    \textbf{1.} \\
    this function is odd, because the function can be rotated $180^\circ$ around 
    the origin and the graph will look the same. \\
    \\
    \textbf{2.} \\
    this function is neither even nor odd because it can not be mirrored over the y-axis 
    nor rotated around the origin to produce the same graph. \\
    \\
    \textbf{3.} \\
    this function is even, because it can be mirrored over the y-axis 
    and the graph will look the same. \\
    \\
    \textbf{4.} \\
    this function is odd, because the function can be rotated $180^\circ$ around 
    the origin and the graph will look the same. \\
    \\
    \textbf{5. $f(x) = 3x^2 + 4$} \\
    \begin{math}
        f(x) = 3x^2 + 4 \\
        f(-x) = 3(-x)^2 + 4 \\
        \phantom{f(-x) } = 3x^2 + 4 \\
        -f(x) = -(3x^2 + 4) \\
        \phantom{-f(x) } = -3x^2 - 4 \\
        f(-x) = f(x) \\
        \therefore \text{the function is even because $f(-x) = f(x)$.} \\
    \end{math}
    \\
    \textbf{6. $f(x) = -2x + 5$} \\
    \begin{math}
        f(x) = -2x + 5 \\
        f(-x) = -2(-x) + 5 \\
        \phantom{f(-x) } = 2x + 5 \\
        -f(x) = -(-2x + 5) \\
        \phantom{-f(x) } = 2x - 5 \\
        \therefore \text{the function is neither even nor odd because $f(-x) \neq f(x)$ or $-f(x)$.} \\
    \end{math}
    \\
    \textbf{7. $f(x) = 2x^2 + 3x$} \\
    \begin{math}
        f(x) = 2x^2 + 3x \\
        f(-x) = 2(-x)^2 + 3(-x) \\
        \phantom{f(-x) } = 2x^2 - 3x \\
        -f(x) = -(2x^2 + 3x) \\
        \phantom{-f(x) } = -2x^2 - 3x \\
        \therefore \text{the function is neither even nor odd because $f(-x) \neq f(x)$ or $-f(x)$.} \\
    \end{math}
    \\
    \textbf{8. $f(x) = -3x^3 + x$} \\
    \begin{math}
        f(x) = -3x^3 + x \\
        f(-x) = -3(-x)^3 + (-x) \\
        \phantom{f(-x) } = 3x^3 - x \\
        -f(x) = -(-3x^3 + x) \\
        \phantom{-f(x) } = 3x^3 - x \\
        f(-x) = -f(x) \\
        \therefore \text{the function is odd because $f(-x) = -f(x)$.} \\
    \end{math}



\end{adjustwidth}


\end{document}